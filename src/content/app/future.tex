




\section*{Deployment and Distribution}
\addcontentsline{toc}{section}{Deployment and Distribution}

\subsection*{Packaged Builds}
% \addcontentsline{toc}{subsection}{Packaged Builds}

The project aims to streamline deployment with packaged builds for easier distribution. Plans include packaging for AUR (Arch User Repository) and as a PyPi (Python Package Index) library.

\section*{Module Packaging}
% \addcontentsline{toc}{section}{Module Packaging}

Each module within the application should be independently packaged to facilitate modular use and distribution through PyPi.

\subsubsection*{Current Progress}
% \addcontentsline{toc}{subsubsection}{Current Progress}

Early builds are available for testing via the following command:
\[
\text{pip install -i https://test.pypi.org/simple/ game-genie}
\]
Support for pip installation is temporarily paused pending further stabilization of the application.

\section*{Performance Enhancement}
% \addcontentsline{toc}{section}{Performance Enhancement}

\subsection*{Evaluation and Optimization}
% \addcontentsline{toc}{subsection}{Evaluation and Optimization}

Efforts are ongoing to evaluate and optimize the application's performance, focusing on runtime efficiency and responsiveness.

\subsubsection*{Speed Analysis}
% \addcontentsline{toc}{subsubsection}{Speed Analysis}

Initial benchmarks indicate satisfactory performance within current scope. Future optimizations will target critical execution time areas.

\subsubsection*{Optimization Strategies}
% \addcontentsline{toc}{subsubsection}{Optimization Strategies}

Proposed strategies include algorithmic improvements, caching mechanisms, and leveraging parallel processing capabilities.


\section*{Conclusion}
% \addcontentsline{toc}{section}{Conclusion}

These planned enhancements are designed to elevate the application's functionality, performance, and usability. By focusing on deployment, optimization, and integration, the project aims to deliver a more robust and efficient toolset for its users.


\pagebreak

\subsection*{Integration with Web Scraper}
\addcontentsline{toc}{section}{Integration with Web Scraper}

\subsection*{Enhancing Machine Learning Capabilities}
% \addcontentsline{toc}{subsection}{Enhancing Machine Learning Capabilities}

Integrating a web scraper component enhances the application's data acquisition capabilities for machine learning tasks.

\section*{Web Crawler: Inner Workings}

% \addcontentsline{toc}{section}{Integration with Web Scraper}
The web crawler script utilizes Selenium for web browsing and Colorama for output coloring. Below are key functions and their mathematical underpinnings:

% \subsection*{Extracting Product Details}

\begin{lstlisting}[language=Python, caption={Function to Retrieve Product Details}]
def getProductDetails(driver):
    try:
        productDetails = driver.find_element(By.CLASS_NAME, "product-details")
    except Exception as e:
        print(Back.RED + f"Could not find product details -> {e} ")
        return None
    return productDetails
\end{lstlisting}

% \subsection*{Processing Prices}

\begin{lstlisting}[language=Python, caption={Function to Extract Lowest Price}]
def getLowestPrice(driver):
    productDetails = getProductDetails(driver)
    if productDetails is None:
        print(Back.RED + f"Could not get product details")
        return -1
    try:
        lowestPriceText = productDetails.find_element(By.XPATH, "//*[contains(@itemprop, 'lowPrice')]")
        lowestPrice = getFloat(lowestPriceText.text)
    except Exception as e:
        print(Back.RED + f"Lowest Price could not be found -> {e} ")
        return -1
    return lowestPrice
\end{lstlisting}

% \subsection*{Mathematical Formulas}

% The function \texttt{getFloat(text)} employs regular expressions to parse numerical values from text, using the following formula:

% \[
% \text{numbers} = \text{re.findall(r'\textbackslash d+', text)}
% \]

% This extracts digits from the text, converting them into a floating-point number.

\subsection*{Conclusion}

By integrating Selenium for web automation and leveraging mathematical parsing techniques, the web crawler efficiently gathers product data from URLs provided as command-line arguments.





\pagebreak





\section*{Web Integration with Flask}
\addcontentsline{toc}{section}{Web Integration with Flask}

\subsection*{Flask Integration}
% \addcontentsline{toc}{subsection}{Flask Integration}

To enable access to the game engine over the internet, integrating a web server using Flask, a micro web framework for Python, is proposed. This allows users to interact with the engine remotely without requiring a local copy.

\subsubsection*{API Endpoints}
% \addcontentsline{toc}{subsubsection}{API Endpoints}

Flask will be used to create API endpoints that interact with the game engine. These endpoints will handle requests such as starting a new game session, controlling game objects, and retrieving game state information.

\begin{lstlisting}[caption={Flask API Endpoint Example}, language=Python]
from flask import Flask, request, jsonify

app = Flask(__name__)

@app.route('/start_game', methods=['POST'])
def start_game():
    game_id = engine.start_new_game()
    return jsonify({'game_id': game_id})

@app.route('/move_object', methods=['POST'])
def move_object():
    data = request.json
    engine.move_object(data['object_id'], data['position'])
    return jsonify({'status': 'success'})

@app.route('/get_state', methods=['GET'])
def get_state():
    state = engine.get_game_state()
    return jsonify(state)

if __name__ == '__main__':
    app.run(debug=True)
\end{lstlisting}




\pagebreak







\subsection*{Database Integration}
% \addcontentsline{toc}{subsection}{Database Integration}

For persistent data storage and management, integrating a database with the Flask application is crucial. SQLAlchemy, a SQL toolkit for Python, facilitates database interactions.

\subsubsection*{Database Models}
% \addcontentsline{toc}{subsubsection}{Database Models}

Define database models to organize and store essential information such as user profiles and game sessions.

\begin{lstlisting}[caption={SQLAlchemy Database Models}, language=Python]
from flask_sqlalchemy import SQLAlchemy

db = SQLAlchemy()

class User(db.Model):
    id = db.Column(db.Integer, primary_key=True)
    username = db.Column(db.String(80), unique=True, nullable=False)
    password = db.Column(db.String(120), nullable=False)

class GameSession(db.Model):
    id = db.Column(db.Integer, primary_key=True)
    user_id = db.Column(db.Integer, db.ForeignKey('user.id'), nullable=False)
    state = db.Column(db.Text, nullable=False)
\end{lstlisting}









\pagebreak



\section*{Integration of Existing OpenGL Solutions}
% \addcontentsline{toc}{section}{Integration of Existing OpenGL Solutions}

This project aims to leverage established OpenGL libraries and tools to enhance graphics rendering and performance. Integrating these solutions will streamline development and extend the capabilities of the game engine.

\subsection*{GLee}
% \addcontentsline{toc}{subsection}{GLee}

GLee simplifies OpenGL extension management and ensures compatibility across different platforms without manual effort. By automatically setting up entry points for OpenGL extensions, GLee can streamline the integration of advanced OpenGL features into the project, reducing development time and enhancing cross-platform support.

\subsection*{GLEW}
% \addcontentsline{toc}{subsection}{GLEW}

GLEW provides efficient methods for checking and using OpenGL extensions and core functionality. By leveraging GLEW, the project can easily incorporate modern OpenGL features and optimizations. Its thread-safe support for multiple rendering contexts ensures robust performance across different rendering scenarios, benefiting both graphics quality and rendering efficiency.

% \subsection*{GLUS}
% % \addcontentsline{toc}{subsection}{GLUS}

% GLUS abstracts hardware and operating system details necessary for graphics programming. By integrating GLUS, the project can utilize its extensive library of functions tailored for OpenGL, OpenGL ES, and OpenVG. This abstraction simplifies graphics programming tasks, enhances portability across different hardware configurations, and facilitates the implementation of advanced graphical effects.

\subsection*{OpenGL Mathematics (GLM)}
% \addcontentsline{toc}{subsection}{OpenGL Mathematics (GLM)}

GLM provides essential mathematical operations and utilities based on the OpenGL Shading Language (GLSL) specification. Integrating GLM allows the project to perform efficient matrix manipulations, vector operations, and geometric computations required for 3D graphics rendering. By using GLM, the project benefits from a standardized and optimized mathematics library tailored for OpenGL-based applications.

\subsection*{libktx}
% \addcontentsline{toc}{subsection}{libktx}

libktx facilitates efficient texture loading and storage using the KTX format, optimized for OpenGL applications. By incorporating libktx, the project can enhance texture management, reduce memory footprint, and improve rendering performance. libktx's support for advanced texture features and seamless integration with OpenGL textures further enhances the graphical fidelity and performance of the game engine.

\subsection*{OpenSceneGraph}
% \addcontentsline{toc}{subsection}{OpenSceneGraph}

OpenSceneGraph exposes OpenGL capabilities while providing extensive features for visual simulation, virtual reality, games, scientific visualization, and modeling. Integrating OpenSceneGraph into the project enhances graphics rendering with advanced rendering techniques, scene management, and multi-threaded processing capabilities. By leveraging OpenSceneGraph's capabilities, the project gains access to a comprehensive toolkit for developing sophisticated graphical applications across various domains.

