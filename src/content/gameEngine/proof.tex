







% In this section, we will inspect the criteria to which a piece of software needs to qualify in order to be considered part of the game engines set.


\begin{center}
\subsection*{Formal Proof}
\textbf{Creation of GG}

\begin{math}
&\text{Let } GG \text{ be the thesis project game engine with } GG.\text{components} = \{M, R, E, FM\}.
\end{math}

\begin{math}
&\text{We know that } \exists Unity \in GameEngines
\end{math}

\begin{math}
Unity.components = \{  \}
\end{math}

More in-depth analysis of Unity's components is done in Field Study chapter.
Right now, we are only interested in the Vector3 and Transform Classes.


\subsubsection*{Demonstration with Unity}

\begin{equation}
\begin{aligned}
&\text{Let Unity be a well-known game engine.} \\
&\text{Unity also includes components } \{M, R, E, FM\}. \\
&\text{Therefore, } \text{isGameEngine(Unity)}.
\end{aligned}
\end{equation}
\end{center}



\subsection*{Formal Conclusion}

\begin{math}
GG \overset{\text{DEFAULT}}{\subseteq} \text{SoftwareEntities} 
\end{math}

\begin{math}
GG \overset{|GG.components| \geq 0}{\subseteq} \text{Frameworks} 
\end{math}

\begin{math}
GG \overset{|GG.components| \geq 0}{\subseteq} \text{GameEngines} 
\end{math}











\begin{equation}
\begin{align}

\label{}
\end{align}
\end{equation}

Let \( S \) be a piece of software, input for a function \( \text{isGameEngine} \) that outputs a binary response.
\[
\text{isGameEngine}(S) = \text{isFramework}(S) \land (\text{S::Components} \subseteq \text{GE::E})
\]


\begin{equation} \label{FrameworkDef}
\begin{align*} 
&GG \in \text{SoftwareFrameworks} 
\rightleftharpoons
&|GG::\text{Components}| \geq 0 
\end{align*}
\end{equation}


where \(\text{GE::E}\) represents the collection of \forall \text{components in } \forall \text{game engines, formulated as:}
\]
\begin{equation} \label{gameEngineDef}
  \text{GE::E} = \bigcup_{G \in \text{GameEngines}} \{(G, C) \mid C \in \text{Components}\}
\end{equation}

\begin{equation} \label{myGameEngineDef}
\begin{align*} 
&\text{Let } GG \in \text{Piece of Software} \\
&\text{Let } GG::\text{Components} = \{M, R, E, FM\}
\end{align*}
\end{equation}


\begin{equation} \label{isGGaGE}
\begin{align*} 
&GG \in \text{GameEngines} 
\rightleftharpoons
&GG::\text{Components} \subseteq \text{GE::E}
\end{align*}
\end{equation}




% \begin{subequation} 
% &GG::M, GG::R, G::E, G::FM \in \text{Components}
% \rightleftharpoons



\text{ denote MathEngine, RendererEngine,
GUI Editor, and File Manager respectively.} \\
\end{subequation}












% \textbf{Given:}
% \begin{equation}
% \begin{aligned}
% &M, R, GUI, FM \text{ denote MathEngine, RendererEngine, GUI Editor, and File Manager respectively.}
% \end{aligned}
% \end{equation}

% \textbf{Statement 1:}
\begin{equation}
\begin{center}
  (&(\exists GG) . \text{isGameEngine}(GG))
  \land

  \land 
  (\text{GG::Components} = \{M, R, GUI, FM\}) \\

&\land (\exists G_i) \, \text{isGameEngine}(G_i) \land (\text{G_i::Components} = \{M, R, GUI, FM\}) \\

&\Rightarrow \text{isGameEngine}(GG)
\end{center}
\end{equation}

\textbf{Statement 2:}
\begin{equation}
\begin{aligned}
&(\exists \text{ Unity}) \, \text{isGameEngine(Unity)} \land (\text{Unity::Components} = \{M, R, GUI, FM\})
\end{aligned}
\end{equation}

\textbf{Proof:}

% \textbf{Statement 1 Proof:}
\begin{equation}
% \begin{aligned}
&\text{Let } GG \text{ and } G_i \text{ be such that } \\
&\text{isGameEngine}(GG) \land (\text{GG::Components} = \{M, R, GUI, FM\}), \\
&\text{isGameEngine}(G_i) \land (\text{G_i::Components} = \{M, R, GUI, FM\}). \\
&\text{Therefore, } \text{isGameEngine}(GG).
% \end{aligned}
\end{equation}

\textbf{Statement 2 Proof:}
\begin{equation}
\begin{aligned}
&\text{Let Unity be such that } \text{isGameEngine(Unity)} \land (\text{Unity::Components} = \{M, R, GUI, FM\}).
\end{aligned}
\end{equation}








\noindent\rule{\linewidth}{0.4pt}


In mathematical notation, the usage of specific game engines in different scenarios can be represented as:

\[
\begin{aligned}
&\text{UnrealEngine} \in \text{PhotoRealisticScenarios} \\
&\text{Unity} \in \text{ClothSimulators} \\
&\text{P5.js} \in \text{PhysicsDemonstrations}
\end{aligned}
\]

\noindent\rule{\linewidth}{0.4pt}

\textbf{Given:}
\begin{align*}
&\text{Let } ME, RE, GUIE, FM \text{ denote MathEngine, RendererEngine, GUI Editor, and File Manager respectively.} \\
&\text{From the code snippets:} \\
&\text{isGameEngine}(ME) \Rightarrow \text{isFramework}(ME) \land (\text{ME::Components} \subseteq \text{GE::E}) \\
&\text{isGameEngine}(RE) \Rightarrow \text{isFramework}(RE) \land (\text{RE::Components} \subseteq \text{GE::E}) \\
&\text{isGameEngine}(GUIE) \Rightarrow \text{isFramework}(GUIE) \land (\text{GUIE::Components} \subseteq \text{GE::E}) \\
&\text{isGameEngine}(FM) \Rightarrow \text{isFramework}(FM) \land (\text{FM::Components} \subseteq \text{GE::E}) \\
\end{align*}

\textbf{To Prove:}
\[
\text{Using or building game engines } GE \text{ is standard industry practice.}
\]

\textbf{Proof:}
\begin{align*}
&\text{Each engine (MathEngine, RendererEngine, GUI Editor, File Manager) is considered a game engine if it satisfies} \\
&\text{the criteria } \text{isGameEngine}(S) \Rightarrow \text{isFramework}(S) \land (\text{S::Components} \subseteq \text{GE::E}). \\
&\text{These engines demonstrate specialized functionalities essential for gaming and diverse applications,} \\
&\text{such as mathematical computations, rendering graphics, user interface design, and file management.} \\
&\text{Their existence and usage across industries highlight the necessity and ubiquity of game engines,} \\
&\text{thus establishing the practice of using or building them as standard in the industry.}
\end{align*}

















