
\chapter*{Benefits of Game Engines}
\addcontentsline{toc}{chapter}{Game Engines}
  Building your own game engine is the standard practice when it comes to big in-house teams. 
  This of course, comes with both advantages and disadvantages.
  The advantages are spaced around the ideas of security and integrity.
  While the disadvantages are mostly cost-related.

  The often solution for the smaller companies remains open-source software.
  There are a few providers on the market already, if names like Unity, Unreal, godot might sound familiar, then maybe also do names like P5.js, Processing or maybe even Coding Train, Sebastian Lague, Code Bullet all the way towards electronics field with Ben Eater. 

  It doesn't matter where in the previous enumeration you lost the references because the following will go more in-depth on the relevant discoveries of each of them.


  \section*{Benefits of building your own game engine}
      \subsection*{Standard industry practice}
          it is common for big companies to build their own in-house game engines and then develop their games on it.
          advantages:     provides competitive edge, security, integrity ...
          disadvantages:  cost, team special for that.

          it is common for smaller sized companies to develop their games/projects on already existing game engines  
          advantages:     already existing reources and docs, community, 
          disadvantages:  dificult to come up with unique style.

          On the following, i want us to analyse some of the game engines that there are. and draw out relevant particularities of each of them.

          for this i have chosen 1 Open Source graphics framework (p5.js), 1 closed software game engine (RAGE) and 1 restricted game engine (UNITY).

          \subsubsection*{RAGE}
              even though this is a closed project and unaccesible to the public, over the years different screenshots and code snippets had been leaked and/or reverse-engineered and i would like us to take a look at some of the more expressive ones.
          \subsubsection*{Unity}
              Even though unity's source code is not accesible to the public, the engine is completly free to use for any individual*.
          \subsubsection*{P5.js}
              This graphics engine is completly free and \href[]{github.com}{open-source}

      \subsection*{Educational purposes}
          i strongly believe that building a game engine had massively improved my abilities.

  \section*{Benefits of integrating machine learning with gaming}

      ml is the new and fancy cool shiny thing that shows promising numbers and gets ppl hyped and everyone loves it and it must be implemented into everything that exists.

      game development is no exception.

      \subsection*{Simulating Human Interaction}
          NPCs are important in games.

          NPCs are there to guide the player and are the projection the game designers into the game world. 

          Because of this, is is really important that npcs have fluid dialogue and dont break the illusion of choice too easily.

          Current solutions imply using dialogue trees.
          
          they can still feel rough on the edges. and the illusion can be broken easily when u have to decide from a set of predefined dialogue choices.

          the imersiveness of games could greatly improve if ml were to be implemented on top of this already existing dialogue tree solution.

          Such solutions have already been experimented with, in the following i will present the findings of 3 other papers that use machine learning to improve npc dialogue and interaction.
          Two of the following are solutions for human-to-ai dialogue and one of them simulates ai-to-ai.
          \subsubsection*{Ai interacting to Ai}
              % TODO: Refactor and fact-check with paper.
              One paper that i found extremily fascinating was \href[]{google.com}{TITLE} by AUTHOR. They created an environment that allowed ml agents to communicate to one another. One of the most exciting outcomes was that one agent organised a birthday party and proceeded to invite other ml agents to the party. In the following i will briefly go over the implementatation design for one agent:
              
              % there were cool images of tables for the datastructures used. they had a timetable and there was a complex way for managing memories and long-time storage of relevant info. 

          \subsection*{Ai interacting to humans}
              Another paper that highlights machine-learning agents interacting in human-like behaviour is \href[]{google.com}{TITLE} by AUTHOR. This team even offers multiple solutions for implementing such agents in popular environments such as Unity or ??.

              One popular demo of their plugin? is the game \href[]{google.com}{GAMENAME}. 
              Game that illustrates a scenario where the player is a detective and has to figure out a case, with the added twist that comunicating with any of the non-playable-characters (NPCs) is made through the microphone and with openai dialogue. 
              
              There is also a mod for the popular game Skyrim that allows the player to have fluid dialogue with any in-game character. 
      
      \subsection*{Out-performing Human Performance}
          popular youtuber Code Bullet has a series where he "solves" games using AI models. He usually uses neural-networks for his solutions. One recent such video is where he programmed a JUMP KING ml.
          
          There are chess bots being developed that use machine learning in an attempt to "solve" the game of chess. So it is clear to say there is is a lot of incentivise towards acomodating machine learning algorithms into games.
