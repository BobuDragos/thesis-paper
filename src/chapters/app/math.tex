
\chapter*{Mathematical Framework}

This mathematical framework is very important because it assures that the phigs requirements can be fulfiled and how to.

\section*{Homogenous Representation}

Intuitively, for storing a 3D point one might think about using a vector of length 3 ( maybe call them x, y, z ) and have a good day.
Well, this datastructure is good \emph{enough} for most cases.
There is, of course, one little edge case in one of the things we must be able to do with this DataStructure that would benefit of storing a 3D point as a 4D Vector (x, y, z, \textbf{\emph{w}}).

\textbf{Please do not be afraid of the \emph{w}.  Since \emph{w} is defined by a pretty straight-forward formula.} 

w=1, if point and 
w=0, if arrow

% TODO: MATH FORMULA FOR w = 1, if point w = 0, if arrow

\subsection*{Vector Multiplication}
\subsection*{Matrix Multiplication}

\section*{Coordinate Systems}

There are two main coordinate systems:
\begin{itemize}
  \item Cartesian Coordinates
  \item Polar     Coordinates 
\end{itemize}

Formulas for converting from one to another:
\begin{itemize}
  \item from Polar     to Cartesian
  \begin{itemize}
    \item cos(x)
    \item sin(x)
  \end{itemize}
  \item from Cartesian to Polar
  \begin{itemize}
    \item x
    \item y
  \end{itemize}
\end{itemize}


In the following, we will take a look over the possible operation that are extensively used:

In most of the cases, the equasions are really easy.

\subsection*{Translation}
\subsection*{Scalation}
\subsection*{Rotation}
\subsubsection*{Euler-Lock}
And it would've been all so easy if it weren't for you! 

Euler Lock is a problem that ocurs when we try to use euler coordinates in rotation aplications. This problem can become extremily dangerous when solving robotics solutions where you can't afford to ???

\subsubsection*{Quaternions}
For eliminating the euler-lock problem, quaternions are used. Quaternions are ??

Formulas:







