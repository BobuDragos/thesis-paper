

\chapter*{Field Study - Software}
\addcontentsline{toc}{chapter}{Software}
  This chapter accepts the embedded solutions as they are and develops solutions that step forward. 
  The usual philosophy is developing abstraction layers.

    \section*{OPENGL - Motivation}
    % \addcontentsline{toc}{section}{opengl}
        There are a few options when choosing a graphics abstraction solution. The most popular in the game development industry are Vulkan and DirectX. DirectX is more appropriate when it comes to Windows-specific optimizations, while Vulkan profits from Low-level control and performance and performs better at High-performance applications with multi-threading.

        The only disadvantage to both those solution is that neither of them is as documented as opengl. Also, opengl is more popular in the educational/academic space and felt like the more appropriate choice.

    \section*{OPENGL - Features}
        OpenGL's extensive documentation comes with both positives and negatives. Being one of the oldest solution to this problem it had seen multiple refactoring stages throught the years, this is best observed when realising there is a new revision of the opengl superbible released every couple of years. Each presenting the usual "How-to" projects and also acting as an update journal.

        \pagebreak

        In the following i will briefly talk about popular opengl features.

